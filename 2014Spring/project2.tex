\documentclass{article}  

\usepackage{booktabs}
\usepackage{hyperlatex}
\usepackage[margin=1.0in]{geometry}
\usepackage{color}

\htmldepth{1}
\htmltitle{CS 150}
\htmladdress{lusth@cs.ua.edu}
\htmlcss{lusth.css}

\T\setlength\parskip{6 pt}
\T\setlength\parindent{0 in}

\W\newcommand\sc\bf

\title{CS150 Honors: Project 2\\
\date{Revision Date: \today}}

\author{Michael G. Raines (modified by John C. Lusth)}

\begin{document}

\maketitle

\thispagestyle{empty}

\W\subsubsection*{\xlink{Printable Version}{project2.pdf}}
\W\htmlrule

\section*{Preamble}

This is your third C assignment.  You may develop your code anywhere,
but you must ensure it runs correctly when compiled by {\it gcc}
on a Linux or Unix system
before submission.
You should read up to and include the textbook chapter
\xlink{Lists and Loops}{../book/index_20.html}
before starting this project.

%Spring 2011

\section*{Logging in the Cloud} 

\begin{quote}
When a jar is broken, the space that was inside\\
Merges into the space outside.
-- 
{\it Avadhuta Gita}
\end{quote}


{\it Welcome to LogSoft}!
Here at Logsoft we specialize in designing and maintaining logging software.

What does logging
software do, you ask?
As your computer or server goes about doing various things,
we would like to keep track of who is doing what and at what time.
In order to do this, we write 
information concerning the commands people run
to a file as the system is running.

Because of our impending move to the cloud, we will need to join 
logs from individual servers to one master log.
Your task is to prototype a program that
merges two individual logs into a single log, sorted
by timestamps. Because the individual logs are also
sorted by timestamp, you should be able to merge them
without resorting to sorting the final result.

Here is an example of a logfile:

\begin{verbatim}
   @ 2014 2 14 00:03:01 Mary "login" 0.01
   @ 2014 2 14 02:06:12 Matt "login" 0.01
   # concurrent logins: 20
   @ 2014 2 14 07:12:05 Mary "cd ~/cs150/projects" 0.01
   @ 2014 2 14 12:33:52 Steven "firefox&" 0.13
   @ 2014 2 14 12:33:52 Steven "load http://yahoo.com" 0.01
   @ 2014 2 14 14:42:27 Steven "exit" 0.00
\end{verbatim}

The format of a single record in a logfile is:

\begin{center}
\begin{tabular}{ccc}%
\T\toprule
    {\sc field} & ~~~~~~~~~~~ & {\sc type} \\
\T\midrule
    {\it type}       & & character\\
    {\it year}       & & integer\\
    {\it month}      & & integer\\
    {\it day}        & & integer\\
    {\it timestamp}  & & string\\
    {\it user}       & & string\\
    {\it command}    & & quoted string\\
    {\it op cost}    & & real\\
\T\bottomrule
\end{tabular}
\end{center}

If the type field is a \verb!#! character, the entire line is a comment.

Your task is to create a program named {\it logmerge.c}
that takes, as input, two
logfile names on the command line.
The program reads the individual logs,
then writes out the merged records
plus some information about the total costs incurred by
the users found in merged logfile.

\section*{I/O}

Suppose file {\it log.1} has the records:

\begin{verbatim}
   @ 2014 2 14 00:03:01 Matt "login" 0.01
   # server down
   @ 2014 2 14 02:06:12 Mary "login" 0.01
   @ 2014 2 14 17:12:05 Mary "cd ~/cs150/projects" 0.01
\end{verbatim}

and file {\it log.2} has the records

\begin{verbatim}
   @ 2014 2 13 12:33:52 Boris "firefox&" 0.13
   @ 2014 2 14 12:33:52 Boris "load http://yahoo.com" 0.01
   @ 2014 2 15 03:42:27 Natasha "exit" 0.00
\end{verbatim}

Then running the program should give the following output:

\begin{verbatim}
    $ logmerge log.1 log.2
    @ 2014 2 13 12:33:52 Boris "firefox&" 0.13
    @ 2014 2 14 00:03:01 Matt "login" 0.01
    @ 2014 2 14 02:06:12 Mary "login" 0.01
    @ 2014 2 14 12:33:52 Boris "load http://yahoo.com" 0.01
    @ 2014 2 14 17:12:05 Mary "cd ~/cs150/projects" 0.01
    @ 2014 2 15 03:42:27 Natasha "exit" 0.00
    # total cost: $0.17
\end{verbatim}

Note that the lines beginning with \verb!#! are comments, which are
removed when reading in records.

The last line of the merged output should be a comment
giving the total cost
of all operations. This information is gleaned from the
last field, the {\it op cost}, in the set of records.
The total cost is the sum of
all the individual op costs.

Should your program be called with fewer than the required number
of arguments, it should generate a meaningful error message and then
exit.

\section*{Command-line Arguments}

Here is a short C program that checks if there are three 
command-line arguments beyond the name of the file.
If so, the program prints out the three arguments:

\begin{verbatim}
    #include <stdio.h>
    #include <stdlib.h>

    int
    main(int argc,char **argv)
        {
        // check to see if there are three arguments
        // look at the length of sys.argv
        // argv holds the name of the program plus the arguments

        printf("the program name is %s\n",argv[0]);
        if (argc == 4)
            {
            printf("argument 1 is %s\n",argv[1]);
            printf("argument 2 is %s\n",argv[2]);
            printf("argument 3 is %s\n",argv[3]);
            }
        else
            {
            printf("incorrect number of arguments\n");
            exit(1):
            }
        return 0;
        }
\end{verbatim}

\section*{Requirements}

You must place your {\it main} function in the file {\it logmerge}.c
and all your other functions in {\it logger.c}.
You must place all the function
prototypes derived from the functions in {\it logger.c} in {\it logger.h}.
The {\it logmerge.c} file should include the {\it logger.h} file,
NOT the {\it logger.c} file.

You must use the scanner for this project:

\begin{verbatim}
    wget troll.cs.ua.edu/cs150H/book/scanner.c
    wget troll.cs.ua.edu/cs150H/book/scanner.h
\end{verbatim}
    
Only {\it logger.c} should include {\it scanner.h}.

To compile your final program, use the command:

\begin{verbatim}
    gcc -Wall -g -o logmerge logmerge.c logger.c scanner.c -lm
\end{verbatim}

You must also define and use
the following functions:

\begin{itemize}
\item
         a function to read an individual record given a scanner -- this
         function must be used to read both logfiles by passing in
         a different scanner -- the function returns a single record --
         the function should skip over comments -- more than one
         comment in a row may appear

\item
     a function to read a logfile into a table -- this function
     should be passed the name of the file to read, opening up
     a scanner to read it -- the function should make use of
     the function that reads records -- it should return a table
     (i.e. an list of records)

\item
     a function that determines which of two records has the
     earlier date and timestamp -- it should return true if
     the first record is earlier than the second and false otherwise

\item
     a function to merge two tables -- it returns the a new table
     holding the merged records -- it makes use of the function that
     decides which of two records has an earlier timestamp
\end{itemize}

\section*{Stepwise Refinement}

This program implements the merge pattern, using records, to combine the
individual logfiles.

\begin{enumerate}
\item
    Write a program named {\it printRecord.c} that reads in a single
record from the first logfile, prints it, and exits.

\item
Write a program named {\it printRecords.c} that loops over all
records in the first logfile, reading in a single record and printing
it before reading in the next record.

\item
Write a program named {\it printAllRecords.c} that does what 
{\it printRecords.c} does and then
does the same for
the second logfile.
You should not write any new functions
to do this.

\item
Write a program named {\it printTable.c} that loops over all
records in the first logfile, reading in a single record and storing
it in a table. After all records have been stored in the table,
loop through the table and print each record in turn.

\item
Write a program named {\it printBothTables.c}
that does what {\it printTable.c}
does and then does the same for the second logfile.
You should not write any new functions
to do this.

\item
Write a program named {\it merger.c} that reads both logfiles into a
table and then merges the two tables into a third table. It should then
print all the records in the third table.

\item
Write the final version of your program; it does what {\it merger.c} does
and then loops through the records accumulating the op costs. It then
prints the total.

\end{enumerate}

\section*{Crafting a Makefile}

You will receive 10 extra credit points for a completely correct and
functioning makefile. Your makefile
must be named with a lowercase initial ``em'' and should be structured as
outlined in the {\it module} activity. You must have a {\it clean} rule.

\section*{Compliance Instructions}

To make sure that you have implemented your
program correctly,  create two files, one named {\it alpha} and one
named {\it beta}. Fill these files with records, making sure one
of the files has at least two comments in a row. Then run the
command:

\begin{verbatim}
    $ gcc -Wall -g -o logmerge logmerge.c logger.c scanner.c -lm
    $ logmerge alpha beta
\end{verbatim}

This test should yield the expected output.
\color{red}
If your code fails with a runtime error while running this test,
then you will
receive a zero for this assignment.
\color{black}
Note that your output does not have to be correct for your program
to be graded, only that it not fail.
Now run this command:

\begin{verbatim}
    $ logmerge alpha
\end{verbatim}

For this test, the program should generate a meaningful error message
(not an index out of range error). Now run the command:

\begin{verbatim}
    $ logmerge
\end{verbatim}

The results of this test should be the same as the previous test.

Programs submitted that generate compilation warnings will receive at
least a 5 point deduction.

\section*{Challenges}

Add an additional command-line
argument; if present, it is the name of the file in which to write
the merged logfile. If missing, the merged logfile should be
written to {\it stdout}, as before.
Using the example from above, if this command is given:

\begin{verbatim}
    $ logmerge log.1 log.2 log.m
\end{verbatim}

then the same output that was written to {\it stdout}
is written to the file {\it log.m} instead.
In fact, these two commands should be equivalent:

\begin{verbatim}
    $ logmerge log.1 log.2 > log.m
    $ logmerge log.1 log.2 log.m
\end{verbatim}

The former redirects the output of the {\it logmerge} program from the
console to the file {\it log.m}. The latter writes the output directly
to {\it log.m}.

Use structures to store records and linked lists to store a set of
tables.

\section*{Submission Instructions}

Change to the directory containing your assignment.  Do an
{\it ls} command to make sure you are in the correct place.
You should see, at least, your {\it logmerge.c} file as well
as your programs written as part of the stepwise refinement.
Extra files are OK. Submit your program like this:

\begin{verbatim}
    submit cs150 YYYY project2
\end{verbatim}

Remember to replace \verb!YYYY! with your instructor name.

\section*{Due Date}

The due date for this assignment can be found on the class
\xlink{schedule}{../schedule.html}.

\end{document}
