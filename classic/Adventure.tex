\documentclass[12pt]{article}
\usepackage{hyperlatex}

%\usepackage{enumitem}
\usepackage{ulem}
\usepackage[compact,medium,sc]{titlesec}
\usepackage{booktabs}
\usepackage[margin=0.5in]{geometry}
\usepackage[pdftex]{graphicx}
\usepackage{multirow}
\usepackage{pslatex}
\usepackage{color}

% paragraph style

\setlength\parskip{10pt}
\setlength\parindent{0in}

\W\newcommand{\sf}{\bf}

\htmltitle{CS100: Project 4}
\htmldepth{1}
\htmladdress{\xlink{lusth@cs.ua.edu}{mailto: lusth@cs.ua.edu}}
\htmlcss{lusth.css}

\title{CS100: Project 4\\
\date{Revision Date: \today}}

\begin{document}

\maketitle

\W\xlink{Printable Version}{project4.pdf}

\section*{Preamble}

\begin{quote}
{\it The game is afoot!}
-- Sherlock Holmes
\end{quote}

The 3-D, immersive, multiplayer video games you are familiar with had
a humble beginning, the text-based adventure game called
{\it Colossal Cave}. Even with its primitiveness, the game is surprisingly
addictive.

You can retrieve a
version of this game with the command:

\begin{verbatim}
    wget troll.cs.ua.edu/cs100/projects/advent.tgz
\end{verbatim}

Unpack the tarball and follow the instructions for compiling
and running the game in the {\it README} file.

Just like you started your adventure in programming with a hello, world
program, you will start your adventure in game programming
by writing a simple, 
text-based adventure game modeled after
{\it Colossal Cave}.

\section*{Approach}

A good way to implement a game of this sort is
to make it as data-driven as possible. A purely data-driven
game is what is called a {\it game} {\it engine},
with text files describing the look and feel, as well as the
actions, of a particular game. A different set of text files
would be used to describe a different game.

You are to implement your game as data-driven as possible
(modulo the fact that you are in an introductory class).

\section{Data structures}

The game is composed of a set of locations,
an adventurer.
and a set of logic functions (one for each location).

\subsection{Locations}

Locations are to be read in from a file that
describes the connections. Here is a possible location description
in such a file:

\begin{verbatim}
name great_chamber
    longer
       "You are in a great chamber. Stalactites hang down, almost touching
        the stalagmites growing up from the cavern floor. There are so many
        glow-worms on the walls, it is too bright to see much of anything."
    shorter
       "You are in the great chamber; it is too bright to see much of anything."
    north maze_1
    south muddy_passage
    logic great_chamber_logic
    item key
    item lantern
    item hat
    ;
\end{verbatim}

Each location is described by a series of attribute-value pairs; the first
pair has the attribute {\it name} followed by the name of the location,
the value {\it great\_chamber}.
The order of the attribute-value pairs and whether or not all attributes must
be present and in a particular order is up to you.

Each location can be modeled as a structure.
Here is a possible structure definition consistent with the above
location description:

\begin{verbatim}
    typedef struct location
        {
        char *name;
        int visited;
        char *longer;
        char *shorter;
        char *north;
        char *south;
        char *east;
        char *west;
        char *logic;
        char *items[MAX_ITEMS];
        } Location;
\end{verbatim}

When reading in a single location, your program should dynamically
allocate a structure and fill it in with the values read.
Code like the following might suffice:

\begin{verbatim}
    loc = malloc(sizeof(Location));
    attr = readToken(fp);
    if (feof(fp)) return 0; //no more locations
    while (strcmp(attr,";") != 0) 
        {
        if (strcmp(attr,"name") == 0)
            loc->name = readToken(fp);
        else if (strcmp(attr,"longer") == 0)
            loc->longer = readString(fp);
        else if (strcmp(attr,"item") == 0)
            addItemToLocation(loc,readToken(fp));
        else if (strcmp(attr,"north") == 0)
            loc->north = readToken(fp);
        ...
        attr = readToken(fp);
        }
\end{verbatim}

The entire set of locations
(known as the {\it game} {\it world})
can be modeled as a linked list.
Each node in
the list has a value pointer that
corresponds to a location in the game world:

\begin{verbatim}
    typedef struct node
        {
        Location *place;
        struct node *next;
        } Node;
\end{verbatim}

Your game should read in a file of locations,
allocate a structure for each location, 
and store the structures in a linked list.
The variable pointing to the linked list can be
a global variable. If the global is named {\it world}, for example:

\begin{verbatim}
    Node *world;
\end{verbatim}

then place this line in all modules that want to access the global:

\begin{verbatim}
    extern Node *world;
\end{verbatim}

You must have at least ten locations in your game. So {\it world} would
need to point to a list of length ten or more.

\subsection{The adventurer}

The adventurer is also modeled with a structure. Such a structure
might look something like this:

\begin{verbatim}
    typedef struct adventurer
        {
        int score;
        int strength;
        char *items[MAX_ITEMS];
        } Adventurer;
\end{verbatim}

The {\it items} array is an optional part of the game and
is used to collect objects from the locations the adventurer visits.
You should initialize the items array to null pointers. That way,
you can add an item by finding a slot that holds a null pointer and
placing the item there. Removing the item would be as simple as finding
its location and setting that slot to zero.

You may want to give strength points or take away strength points from
the adventurer as the game progresses. Strength points, however,
are entirely optional and you may add other variables to the structure
as you see fit.

\subsection*{Logic functions}

You will need a logic function for each location in the game world.
Here is a simple logic function that just allows moves to a new
location:

\begin{verbatim}
    Location *
    basic_location_logic(Adventurer *a,Location *b,char *verb,char *noun)
        {
        Location *B = b;      //assume next location B is the current one
        b->visited = 1;
        if (strcmp(verb,"north") == 0)
            B = findLocation(b->north);
        else if (strcmp(verb,"south") == 0)
            B = findLocation(b->south);
        else if (strcmp(verb,"east") == 0)
            B = findLocation(b->east);
        else if (strcmp(verb,"west") == 0)
            B = findLocation(b->west);
        else
            printf("I can't do that.\n");

        if (B == 0)
            return b;
        else
            return B; //B should hold the next location
        }
\end{verbatim}

If you make this the logic function of all your locations, you should
be able to move around your world (provided you implement 
{\it findLocation}). Note: the intent of the {\it visited} field
is that the longer description of the location is given upon first visitation
and the shorter description given upon subsequent visitations.

Here is a sample logic function that shows a bit more sophistication:

\begin{verbatim}
    Location *
    great_chamber_logic(Adventurer *a,Location *b,char *verb,char *noun)
        {
        Location *B = b;      //assume next location B is the current one
        b->visited = 1;
        if (strcmp(verb,"what") == 0)
            displayItems(a);
        else if (strcmp(verb,"drop") == 0)
            dropItem(noun,a,b);              //move item from a to b
        else if (strcmp(verb,"say") == 0)
            printf("your words seem to have no effect\n");
        else if (hasItem(a,"sunglasses"))
            {
            if (strcmp(verb,"look") == 0)
                displayLongDescriptionWithItems(b);
            else if (moveIsDesired(verb))
                B = findLocation(b,verb);
            else if (strcmp(verb,"get") == 0)
                getItem(noun,a,b);           //move item from b to a
            else
                printf("I don't understand.\n");
            }
        else if (strcmp(verb,"look") == 0)
            displayLongDescription(b);
        else
            printf("It is so bright, I can't do that.\n");

        return B; //B should hold the next location
        }
\end{verbatim}

In this example, the intent is that the adventurer must have the
sunglasses to get out of the chamber. If the user is carrying sunglasses,
then the adventurer can move about. If the adventurer enters the
chamber without them, then he or she is trapped for eternity.

Also, in our example, the adventurer can only move north and south
(with sunglasses). If the adventurer tries to move, say, east,
the {\it findLocation} function should print an appropriate message and
then return the current location (since the adventurer tried to move
somewhere he or she cannot and therefore must stay in the current location).

Finally, the {\it dropItem} should remove the given object from the
adventures {\it items} lists and add it to the location's items list.

Each of your locations will have an associated logic function. In some
instances, it may be appropriate for locations to share a logic function.
(e.g. the {\it basic\_location\_logic}).

\section*{The game core}

After reading in the set of locations,
start the game by choosing one of the locations as the starting location.
The core of your game is a loop that performs the following actions:

\begin{verbatim}
    location = getFirstLocation();
    while (location != 0)
        {
        //display the current location
        //get a legal utterance
        //find the logic function associated with the current location
        //call the logic function, resesting the location with the return value
        }
\end{verbatim}

You can end the game by having the logic function return zero when a
winning condition is determined:

\subsection*{Getting a legal utterance}

The minimum set of utterances (commands) that your game should understand are:

\begin{verbatim}
north         move to the location to the north
south         move to the location to the south
east          move to the location to the east
west          move to the location to the west
look          describes the current location, what can be found there,
              what can be seen in the four directions, and all information
              about the adventurer (e.g. items held, strength points).
\end{verbatim}

You may find other utterances useful:

\begin{verbatim}
get <object>  pick up an object
drop <object> release an object
say <word>    speak
\end{verbatim}

You may add other commands (and abbreviations for commands)
as you see fit. The utterance should be broken up into verb and noun,
where the first token in the utterance
is considered the verb and the second the noun.
For example, the utterance ``get keys'' would have a verb of
``get'' and a noun of ``keys''. The utterance ''north'' would have
a verb ``north'' and a null noun.

Your game should be rated \verb!eC!, \verb!E!, \verb!E10+!, or \verb!T!
(see \xlink{http://www.esrb.org/esrbratings\_guide.asp}
           {http://www.esrb.org/esrbratings\_guide.asp}).

\subsection*{Finding and calling a logic function}

A logic-function finder would take the name of the logic
function and return a pointer to the logic function with that name.
A function
pointer can then
be called as if it were a defined function name. To do this,
first we make a logic function type:

\begin{verbatim}
    typedef Location *(*Logic)(Adventurer *,Location *,char *,char *);
\end{verbatim}

With this typedef, we can use the name {\it Logic} to define pointers
to logic functions. With the typedef,
the following variable definitions are
equivalent:

\begin{verbatim}
    //create a variable f that can point to a logic function
    Location *(*f)(Adventurer *,Location *,char *,char *);
    Logic f;
\end{verbatim}

We can use a function pointer to find and call a logic function:

\begin{verbatim}
    //create a holder for the name of the logic function
    char *logicName = currentLocation->logic;

    //create a function pointer f; f can point to a logic function
    Logic f;

    //retrieve the logic function and store it in the pointer f
    f = findLogicFunction(logicName);

    //call the logic function through f
    f(adventurer,currentLocation,verb,noun);
\end{verbatim}

Unfortunately in C, there is no easy way to turn a string into
a function. However, an if-else\_if-else chain
can be used to search for the proper function:

\begin{verbatim}
    Logic g;

    if (strcmp(logicName,"basic_location_logic") == 0)
        g = basic_location_logic;
    else (strcmp(logicName,"great_chamber_logic") == 0)
        g = great_chamber_logic;
    else if (strcmp(logicName,"muddy_passage_logic") == 0)
        g = muddy_passage_logic;
    ...
    else
        {
        fprintf(stderr,"INTERNAL ERROR: logic function %s not found\n",logicName);
        exit(-1);
        }

    return g;
\end{verbatim}

Note that the intent of this code
fragment is to store the proper logic function 
in the function pointer named {\it g} and then return the value
of this pointer.

There is a sample program that illustrates using a typedef in this
manner for finding and calling logic functions:

\begin{verbatim}
    wget troll.cs.ua.edu/cs100/projects/typedef.c
\end{verbatim}

Base your code off of that example.

\subsection*{Changing the location}

The {\it findLocation} function returns the new location associated
with a successful
move. Th location found is then returned by the logic function.
The main loop that is controlling the game then sets the current
location to this return value. Of course, if an illegal move is
requested or a move is not requested at all, the logic function
returns the current location and the adventurer stays in the same place

\section*{Step-wise refinement}

Here is one possible path to a working implementation:

\begin{itemize}
\item
Define a function to read in a location. Then, print the details
of the location.

\item
Define a function that reads in a series of locations (into a linked list).
Then, print the details of each location.

\item
Define a series of logic functions. Define a function that takes the
string name of the function and returns a pointer to that function.
Call the returned function to make sure everything works. You can
simplify the logic of the function to a print a simple statement
and to return the current location.

\item
Point all your locations to the {\it basic\_location\_logic} function.
Your game should now allow you to move through your game world.

\item
For extra credit:
revise your logic functions further, one at a time, so that they
have there full logic. Test each location thoroughly with all
legal utterances.
\end{itemize}

\section*{Documentation}

As always,
you should identify the author of any code you submit.  Document any
code that is not obvious to a relatively sophisticated reader.

Provide a file named {\it README} that describes how to play the game.
In the case of incomplete programs, your README should document what
capabilities your program does have.

Provide another file named {\it MAP.jpg} that contains a picture
of the map of your game world. You may hand draw the map if you wish.

Provide a third file named {\it PUZZLE} that contains the contains
the utterances that solve the puzzle and win the game.

\section*{Compliance}

The name of your executable should be {\it game100}.

You must supply a makefile that responds to the following commands:

\begin{verbatim}
    make 
    make clean
    make run
    make puzzle
\end{verbatim}

The first command should build your project while the second should delete
the executable and object files. The third command should run the game
with input from the keyboard:

\begin{verbatim}
    run: game100
        game100
\end{verbatim}

while the fourth command should run the game with the {\it PUZZLE} file, as in:

\begin{verbatim}
    puzzle: game100
        cat PUZZLE | game100
\end{verbatim}

When \verb!make puzzle! command is invoked, the game should run automatically
to its conclusion.

The first rule in your makefile should be the {\it game100} rule.
The {\it run} and {\it puzzle} rules should have {\it game100} as a dependency.

In the case of incomplete programs,
your {\it puzzle} rule
of the makefile
should print approriate
messages, as in:

\begin{verbatim}
    puzzle : game100
        echo The program only prints out the locations
\end{verbatim}

\section*{Grading}

Points will be deducted
for not adhering to the specifications given in this document and the
syllabus. Points will be deducted for bad style, especially unreasonable
amounts of duplicated code, as well as for sloppy formatting, insufficient
or overly verbose documentation, compiler warnings, run-time crashes,
and other such transgressions. You will receive no credit if your program
does not compile.

A small amount of credit (25 points)
will be awarded to a program that can just
read in locations into a linked list and then display the locations
in that list. Deductions described previously apply when appropriate.

Full credit (100 points) will be given if 
the adventurer can move around the
game world. The only logic that needs to be implemented is
for moving to a new location (i.e., no puzzle, no items).
Deductions described previously apply when appropriate.

Exceptionally entertaining implementations that use sophisticated
logic and the use of items to solve a puzzle will receive a bonus of up
to 50 points.

\section*{Submitting the assignment}

To submit your assignment, run a make clean command and then type
the command:

\begin{verbatim}
    submit cs100 YYY project4
\end{verbatim}

where {\it YYY} is your instructor drop box name.

\end{document}
