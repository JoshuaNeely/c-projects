\documentclass{article}  

\usepackage{booktabs}
\usepackage{hyperlatex}
\usepackage[margin=0.5in]{geometry}
\usepackage{color}

\htmldepth{1}
\htmltitle{CS 150}
\htmladdress{lusth@cs.ua.edu}
\htmlcss{lusth.css}

\T\setlength\parskip{6 pt}
\T\setlength\parindent{0 in}

\title{CS 150~~Programming I\\
Project 2}

\date{Revision Date: \today}

\begin{document}

\maketitle

\thispagestyle{empty}

\W\subsubsection{\xlink{Printable Version}{project2.pdf}}
\W\htmlrule


%Fall 2008

\section*{Preamble}
This is your third C assignment.
You may develop your code anywhere,
but you must ensure it
runs correctly when compiled by {\it gcc}
on a Linux or Unix system
before submission.
You should read up to and include the textbook chapter
\xlink{Lists and Loops}{../book/index_20.html}
before starting this project.


\section*{Movie Database} 
How much digital data will the world create and store this year?
Some place that number over 2 zettabyes! That's 21 zeros and is
roughly the amount of storage  of 57.5
billion 32GB Apple Ipads.  Furthermore, the
amount of data created and stored doubles every two years.  How do
we store and access this information? Usually with storage software
known as a {\it database}.  Databases are stored as files, and due
to restrictions on file sizes, sometimes it is necessary to split
a database across multiple files.
The goal of this project is to implement a simple
C database and query system that reads in a database from two files,
merges the records from each database file together, and allows 
the user to perform some basic queries.  The individual
record sets will be ordered by year in descending order.  The merge
pattern will combine the two data sets into one data set
in such a way that the ordering is preserved.

Tasks for your program:

\begin{itemize}
\item
    Display Movies
\item
    Display longest movie
\item
    Display shortest movie
\item
    Display movies earlier than specified year
\item
    Display movies later than specified year
\item
    Display movies with specified rating
\item
    Add Movie
\item
    Delete Movie
\item
    Save database
\end{itemize}


\section*{Program behavior}

Your program should use a menu driven interface that allows the user
to interact with one table in a database.  Once your program is launched,
it should read in the files containing movie records, merge the records together,
and then display the following menu:

\begin{verbatim}
    Welcome to the Movie Database!
    Main menu: 
        1:  display all movies
        2:  display shortest movie
        3:  display longest movie
        4:  display older movies
        5:  display newer movies
        6:  display movies by rating
        7:  add movie (CHALLENGE)
        8:  delete movie (CHALLENGE)
        9:  save database (CHALLENGE)
        0:  quit the program
    Enter an option:
\end{verbatim}

Reading in the menu option should be placed in
a loop. Only when the user enters a ``0'' should your program terminate..

\subsection*{Option 1: Display all movies}

Entering ``1'' will display the entire contents of the movie database.
Based on the test {\it data} file, when you select the first option,
your program should display something similar to:

{\scriptsize
\begin{verbatim}
1 "The Grand Budapest Hotel" "A legendary hotel concierge is framed for murder." 2014 100 R "Ralph Fiennes, F. Murray Abraham" "Wes Anderson" 
2 "A Beautiful Mind" "True story of a math prodigy overcoming schizophrenia" 2011 136 PG-13 "Russell Crowe, Jennifer Connelly" "Ron Howard"
3 "Pulp Fiction" "Two thugs, boxer and crime boss meet their fates" 1995 154 R "John Travolta, Samuel L. Jackson" "Quentin Tarantino"
4 "The Shawshank Redemption" "Innocent man sent to prison for life" 1994 142 R "Tim Robbins, Morgan Freeman" "Frank Darabont"
5 "Tombstone" "Doc joins Wyatt for OK Corral showdown" 1993 130 R "Kurt Russell, Val Kilmer" "George Cosmatos"
6 "Full Metal Jacket" "Marine seen from basic training to Tet offensive" 1987 117 R "Matthew Modine, Adam Baldwin" "Stanley Kubrick"
7 "Blade Runner" "Futuristic detective hunts obsolete androids" 1982 122 R "Harrison Ford, Sean Young" "Ridley Scott"
8 "Alien" "A slimy creature boards a woman's space tanker" 1979 117 R "Sigourney Weaver, Tom Skerritt" "Ridley Scott"
9 "2001: A Space Odyssey" "Supercomputer HAL takes astronauts on ultimate trip" 1968 139 G "Keir Dullea, Gary Lockwood" "Stanley Kubrick"
10 "The Wizard of Oz" "A tornado whisks Dorthy to a magical land" 1939 101 G "Judy Garland, Ray Bolger" "Victor Fleming"
\end{verbatim}
}

\subsection*{Option 2: Display shortest movie}

This option displays the shortest movie in the database. You will need to
search your list of movies to find the shortest one.
Note that your
program will be tested on other databases.

\subsection*{Option 3: Display longest movie}

Like Option 2, but displays the longest movie.

\subsection*{Option 4: Display older movies}

When the user selects this option, the program queries the user for a year:

\begin{verbatim}
    Enter option number: 4
    Display movies older than what year? 1998
    ...
\end{verbatim}

The program then displays movies that are older than the given year.

\subsection*{Option 5: Display newer movies}

Like Option 4:

\begin{verbatim}
    Enter option number: 5
    Display movies newer than what year? 1998
    ...
\end{verbatim}

but movies newer than the given year are displayed.

\subsection*{Option 6: Display movies by rating}

When the user selects this option, the program queries the user for a rating:

\begin{verbatim}
    Enter option number: 6
    Display movies with what rating? R
    ...
\end{verbatim}

The program then displays movies with the given rating.

\subsection*{Option 7: Add a movie}

This is a CHALLENGE option. If you do not implement this option, then
your system should display an appropriate message:

\begin{verbatim}
    Enter option number: 7
    Option 7 has not been implemented
\end{verbatim}

If you implement this option, your program should
prompt the user as follows:

\begin{verbatim}
    Enter option number: 7
    Adding movie... 
        Title: "My Cool Movie!"
        Description: "A movie that I made with some friends"
        Year: 2012
        Length in minutes: 145
        Rating: G
        Main actors: "Gabriel"
        Director: "John Lusth"
    Movie added.
\end{verbatim}

In the example, the user has entered the text \verb!"My Cool Movie"!,
\verb+"A movie that I made with some friends"+, and so on.
Selecting Option 1 again should display the new movie along with the others.
You may place the new movie at the end of the database.
You will need to reallocate the movie array to store the new record.

\subsection*{Option 8: Delete a movie}

This is a CHALLENGE option. If you do not implement this option, then
your system should display an appropriate message:

\begin{verbatim}
    Enter option number: 8
    Option 8 has not been implemented
\end{verbatim}

If you implement this option, your program should
prompt the user as follows:

\begin{verbatim}
    Enter option number: 8
    Delete movie with which ID? 10
    Movie #10 has been deleted.
\end{verbatim}

The deleted movie should not be seen when Option 1 is reselected.

\subsection*{Option 9: Save the database}

This is a CHALLENGE option. If you do not implement this option, then
your system should display an appropriate message:

\begin{verbatim}
    Enter option number: 9
    Option 9 has not been implemented
\end{verbatim}

If you implement this option, your program should
prompt the user as follows:

\begin{verbatim}
    Enter option number: 9
    Save database to what file? saved.db
    Movie database has been saved.
\end{verbatim}

\section*{Starting the project}

Create a directory named {\it project2} that hangs off of your 
{\it cs150} directory. Do your work in that directory.

Your first task is to create a file containing movie database entries.
Each record needs to be contain the following fields:

\begin{verbatim}
MOVIE_NAME DESCRIPTION YEAR LENGTH RATING CAST DIRECTOR
\end{verbatim}

You will be using the {\it scanner} to read this data in, so it is important that
each record in the file contains the same structure.  The type
of scanning functions you use should be based on the type of data stored in each record,
which should be as follows:\\\\
\begin{tabular}[h]{ll}
{\it MOVIE\_NAME} & string\\
{\it DESCRIPTION} & string\\
{\it YEAR} & int\\
{\it LENGTH} & int\\
{\it RATING} & token\\
{\it CAST} & string\\
{\it DIRECTOR} & string\\
\end{tabular}

An example database record could be:
{\scriptsize
\begin{verbatim}
"Alien" "A slimy creature boards a woman's space tanker" 1979 117 R "Sigourney Weaver, Tom Skerritt" "Ridley Scott"
\end{verbatim}
}

Note that there is no requirement for fields of a record to appear on the same line.
This rendering of a database record would be legal as well:

{\scriptsize
\begin{verbatim}
"Alien" 
     "A slimy creature boards a woman's space tanker"
1979 
     117 
R 
     "Sigourney Weaver, Tom Skerritt"
"Ridley Scott"
\end{verbatim}
}

After creating a file containing movie database entries, you are ready to
proceed to the next step, creating a data structure to hold fields of a record.

\section*{Database Movie Structure}

An important aspect of writing any program is to be mindful of how data should
be represented and stored.  Considering the types of data stored in each record of the database,
it would be nice to be able to group that data together in a logical way.  This task can
be accomplished easily with a structure. 
Please read the \xlink{{\it Structures}}{../book/index\_17.html}
chapter of the book before continuing.

The {\it incomplete} structure that you may use to store data is as follows:

\begin{verbatim}
    typedef struct entry
        {
        char *name;
        char *description;
        int year;
        ...
        } Movie;
\end{verbatim}

Complete the structure and place it in a file named {\it support.h}.

We can now read records into the {\it Movie} structure similarly
to how reading a file was accomplished in {\it Project 1}.  The difference
is that in Project 1, you counted the tokens in advance in order to
size the array to store the tokens properly. In Project 2,
you should resize the{\it  Movie} array as needed, while reading
in the records.
Consider the
following {\it incomplete} example for reading a database file and printing
out each record:

\begin{verbatim}
    #include <stdio.h>
    #include <stdlib.h>
    #include "support.h"
    #include "scanner.h"
    
    int
    main(int argc, char **argv)
        {
        int count;
        Movie **entries;
        
        entries = fillMovieArray("someFileName",&count);
        printMovies(entries, count);
        ...
        return 0;
        }
\end{verbatim}

with {\it fillMovieArray} and {\it printMovies} defined in {\it support.c}:

\begin{verbatim}
    Movie **
    fillMovieArray(char *fileName, int *finalSize)
        {
        // initialize stuff (including the array, count, and size)
        ...
        // read the first record
        ...

        while (!feof(...))
            {
            // the read was good, make room for the record if necessary
            ...
            // store the record in the array and increment the count
            ...
            // read the next record
            }
        ...
        // clean things up
        ...
        // return array
        }

    void
    printMovies(Movie **entryArray, int num)
        {
        int i;
        printf("Movie Listings:\n");
        for (i=0; i<num; ++i)
            {
            printf("\"%s\" \"%s\" %d %d %s \"%s\" \"%s\"\n",
                               entryArray[i]->name,
                               entryArray[i]->description,
                               entryArray[i]->year,
                               ...
            }
        }
\end{verbatim}

To read from a file, you are required to use the {\it scanner} module, which
you can retrieve with:

\begin{verbatim}
    wget http://troll.cs.ua.edu/cs150/book/scanner.c
    wget http://troll.cs.ua.edu/cs150/book/scanner.h
\end{verbatim}

Run this command in the same directory as your project files.

\section*{Querying the database}

Once all of the entries have been read into an array of {\it Movie} structures
we can start to implement the features required by this project.
Each query
corresponds to one, or more, loop patterns found in
the \xlink{book}{http://troll.cs.ua.edu/cs150/book/index\_15.html}.

The patterns that you will need to implement are as follows:
\begin{itemize}
\item extreme {\it or} extreme-index
\item filter
\end{itemize}

The query system follows the basic {\it read pattern}:

\begin{verbatim}
    void
    query(Movie **db, int count)
        {
        int option;

        displayMenu();

        // read the option
        option = readInt(stdin); //read from the keyboard
        while (option != 0)
            {
            if (option == 1)
                printMovies(db,count);
            else if (option == 2)
                printf("option 2 not implemented yet\n");
            else
                {
                printf("option not understood\n");
                displayMenu();
                }
            option = readInt(stdin);
            }
        }
\end{verbatim}

Fill in options as you proceed.

\section*{Program organization}

Name your project {\it moviedb.c}, and create another file, {\it support.c},
along with an appropriate header file named {\it support.h}. 
The support files will contain all of the helper functions needed by your program.  Your
main function should only call functions that are found in {\it support.c}.

Your main should read the database files and count the number of records, via helper functions.
It should then dynamically allocate arrays to store the records in each file and reread the files,
again via helper functions.  Next, it should merge the records and then prompt the user for queries.

%These are some of the support functions you will need:
%\begin{itemize}
%\item a function that tokenizes a database file, returning the count of records - this function implements the counting pattern
%\item a function that, given a count, allocates and returns an array of records with the given size
%\item a procedure that, given a file and an array, tokenizes the database file, filling the given array
%\item a function that, given two arrays of records, merges the records together and returns the resultant array
%\item a procedure that displays a menu of options and prompts for a choice - this function will call the appropriate helper function based on the user choice
%\item a function that, given an array of records and a year, returns the oldest record
%\item a function that, given an array of records and a year, returns the newest record
%\item a function that, given an array of records and a year, returns an array of records older than the specified year
%\item a function that, given an array of records and a year, returns an array of records newer than the specified year
%\item a function that, given an array of records and a rating, returns an array of records with the specified rating
%\item a procedure that, given an array of records, prints out each record, one record per line
%\end{itemize}

Place your {\it main} function in a file {\it moviedb.c}. Place other functions/procedures in a file named {\it support.c} and their function signatures into {\it support.h}. Remember to include {\it support.h} in {\it moviedb.c} with the command:

\begin{verbatim}
    #include "support.h"
\end{verbatim}

The database filenames should be read in using {\it command line arguments}, as follows:

\begin{verbatim}
    $ moviedb records0 records1
\end{verbatim}

where {\it records0}, and {\it records1} are the database filenames.  Additionally, each
file will be individually sorted in decreasing order by year.

\section*{Stepwise refinement}

Never try to write an entire program in one go. Always implement a little bit of what you need to do and test that bit thoroughly with a specialized main function. This technique is known as stepwise refinement.

\subsection*{Level 0}
Define a function that reads a single record.
Write a program named {\it level0.c} that tests this function.

\subsection*{Level 1}
Define a function that reads one of the database files and
a procedure to display them.
Copy the {\it level0.c} program to {\it level1.c}.
Modify {\it level1.c} so that it tests these routines.

\subsection*{Level 2}
Copy the {\it level1.c} program to {\it level2.c}.
Modify the {\it level2.c} program to read and display both database files.
You should not write any
new functions, instead re-use the functions and procedures
that you have already defined.

\subsection*{Level 3}
Write a function that merges two arrays of records. This function
should take four arguments, the first array and its count and
the second array and its count.
The function should allocate (and eventually return) the destination
array, which holds the set of merged records.
The size of the destination array can be computed from the 
counts of the incoming arrays.
Note, if the records in the incoming array are sorted, the
records in the destination array should be sorted as well.
To copy a single record from an incoming array to the destination
array, code similar to the following can be used:

\begin{verbatim}
    out[spot] = in1[index1];
    ++spot;
    ++index1;
\end{verbatim}

Copy the {\it level2.c} program to {\it level3.c}.
Modify {\it level3.c} to test the merging.

\subsection*{Level 4}
Write a function that receives an array of merged records, and returns the shortest movie in the array.  Test this function by printing out the
resulting record and verifying that it is correct by inspecting the database manually.  If it is not correct, revisit the {\it extreme} pattern
from the book and try again.  Write a function for retrieving records newer than a specifed year, and test it similarly.
Copy the {\it level3.c} program to {\it level4.c}.
Modify {\it level4.c} to test the new function.

\subsection*{Level 5}
Copy {\it level4.c} to {\it level5.c}.
Integrate the query system into {\it level5.c}
and test the display of the shortest
movie via the query system.

At this point, copy {\it level5.c} to {\it moviedb.c} and add the remaining options to your program. The more
options you implement, the higher your grade. If your program can correctly
handles options 0 and 1, you will receive, at a minimum, a score 60/100.

For this project you {\bf are} required to submit each level.  Only those students who submit each level, including
related testing code will receive full credit.

\section*{Submission Instructions}

Change to the {\it project2} directory containing your assignment.  Use the
{\it ls} command in your projects directory. You should see something like this:

\begin{verbatim}
    moviedb.c   scanner.h   scanner.c   support.h   support.c   records0   records1
\end{verbatim}

Extra files are OK. Submit your project like this:

\begin{verbatim}
    submit cs150 YYY project2
\end{verbatim}

Remember to replace {\tt YYY} with your instructor name followed by class
time.

\section*{Due Date}

The due date for this assignment can be found on the class
\xlink{schedule}{../schedule.html}.

\end{document}
