\documentclass{article}  

\usepackage{booktabs}
\usepackage{hyperlatex}
\usepackage[margin=1.0in]{geometry}
\usepackage{color}

\htmldepth{1}
\htmltitle{CS 150}
\htmladdress{lusth@cs.ua.edu}
\htmlcss{lusth.css}

\T\setlength\parskip{6 pt}
\T\setlength\parindent{0 in}

\W\newcommand\sc\bf

\title{CS 150~~Programming I\\
Project 3: Organization}

\author{John C. Lusth}
\date{Revision Date: \today}

\begin{document}

\maketitle

\thispagestyle{empty}

\W\subsubsection*{\xlink{Printable Version}{project3-org.pdf}}
\W\htmlrule

\section*{Program Organization}

You must follow and implement the functions, procedures,
and programs described in
this document to receive any credit.

The stepwise refinement methodology of the previous programming assignment
exhibited a {\it top-down} approach.  In a top-down approach, the main
function is sequentially refined until the project is finished. For this
project, you should take a {\it bottom-up} approach, in which you start
out by writing the most basic functions and then write higher-level
functions that call these basic functions, and so on.

\begin{description}
\item[Level 0]
    A board can be represented as a two-dimensional array or matrix. A
    location on the board can be represented as a row and a column.
    Define a set of functions and procedures
    for allocating and populating a board,
    given the specifications from the command line.\W\\
    \W~

    \begin{itemize}
        \item One function should allocate an empty board. Two boards will be
            allocated with this function, one which will hold mine information
            (the {\it mine} board) and one which holds the covered/uncovered
            status of each square (the {\it status} board).
        \item A second procedure should place the desired number of mines on
            a given {\it mine} board.
        \item A third function, when given a {\it mine} board and a location,
            calculates the number of mines adjacent to that location.
            It does this by checking each of its eight (or fewer) neighbors
            for mines.
        \item A fourth procedure should place the number of adjacent mines in
            each square in a given {\it mine} board (via the third function).
        \item A fifth procedure should initialize the given {\it status}
            board so that every square is considered covered.
    \end{itemize}

    For the {\it mine} board, you will need to decide on a value to
    represent a mine that is distinguishable from the count of neighboring
    mines. For example, you can't use the number 1 to represent a mine,
    since it could also represent a count. For the {\it status} board,
    you will need a value that represents a covered square and a value
    to represent an uncovered square.

    Place these functions and procedures in a module named {\it game.c}.

    Write a program named {\it level0.c} that thoroughly tests the
    board allocation functions and initialization procedures.
    This program should include the {\it game.h} header and link in
    the {\it game.o} module.

    Have a rule in your makefile that responds to:

\begin{verbatim}
        make level0
\end{verbatim}

\item[Level 1]
    Define a procedure that displays a board. This procedure is given two
    boards, a {\it mine} board and a {\it status} board.  As the board is
    being drawn, the procedure checks the {\it status} board to see if a
    covering mark should be displayed or if the number of adjacent mines
    (found on the {\it mine} board) should be displayed.

    Place this procedure in {\it io.c}.

    Write a program named {\it level1.c} that thoroughly tests your display
    procedure. Your program should print the initial board, manually uncover
    a few squares, print the board, manually uncover all the squares,
    and print the board one final time. This program will need to
    include the {\it game.h} and {\it io.h} header files and link in
    the {\it game.o} and {\it io.o} modules.

    Have a rule in your makefile that responds to:

\begin{verbatim}
        make level1
\end{verbatim}

\item[Level 2]
    Define an pyrotechnic display procedure. At the start, this procedure
    should just print the word BOOM.

    This procedure should be placed in {\it io.c}.

    Write a program named {\it level2.c} that 
    prints the initial board, manually uncover
    a few squares, prints the board again, and calls the
    pyrotechnic display procedure.

    Have a rule in your makefile that responds to:

\begin{verbatim}
        make level2
\end{verbatim}

\item[Level 3]
    Define an ``uncovering'' function that, when given a {\it mine}
    board, a {\it status} board, and a location, uncovers the square on
    the given board. If the square has no adjacent mines, the function
    recursively calls itself to uncover all the neighbors of the square.
    If the function uncovers a square with a mine, it should return
    1. Otherwise, it should return 0.

    Place this function in {\it game.c}.

    Write a program named {\it level3.c} that 
    prints the initial board, calls the uncovering function
    to uncover
    a square that should have a cascading effect, and prints the board again.

    Have a rule in your makefile that responds to:

\begin{verbatim}
        make level3
\end{verbatim}

\item[Level 4]
    Define a ``end of game'' checking 
    function that, when given a {\it mine} board and a {\it
    status} board, returns 0 if all covered squares contain mines and
    1 otherwise.

    Write a program named {\it level4.c} that 
    tests the end-of-game checking function thoroughly.

    Have a rule in your makefile that responds to:

\begin{verbatim}
        make level4
\end{verbatim}

\item[Level 5]
    Define a ``move'' function that, when given a {\it mine} board and a {\it
    status} board, immediately calls the end-of-game checking function.
    If it returns 0, then the this function should return 2.
    Otherwise, this function prompts for a location
    and then uncovers that square
    using the function from Level 3. If the uncovering function returns
    a 0, this function should display the board using the procedure from
    Level 1 and return a 0. If the uncovering function returns a 1, then
    the pyrotechnic display procedure from Level 2 should be called and
    a value of 1 returned.

    Note that a return value of zero means the game should continue. A
    return value of 1 or 2 means the game is over.

    Place this function in {\it io.c}.

    Write a program named {\it level5.c} that 
    tests the three return values of the move function.

\item[Level 6]
    Define a ``game loop'' procedure that implements a classic
    ``reading'' loop for
    running the game.  The initial {\it mine} and {\it status} boards should
    be passed to this procedure.
    Before the loop starts, the move function from Level
    5 should be called and the return value saved. As long as this saved
    return value is 0, the loop should run. The body of the loop reruns
    the move function, resaving the return value.
    When the loop exits, either call the pyrotechnic display procedure
    or a congratulatory message.

    Place this procedure in {\it minesweeper.c}
    along with a 
    {\it main} function. The {\it main}
    function should allocate the game boards
    and then call the game-loop procedure.

    Your final program should consist of {\it minesweeper.c}, {\it game.c},
    and {\it io.c}, plus the header files.

\item[Level 7]
    Refine the procedures
    from Level 1 and Level 2 to make a more spectacular game.
    Making system calls with the clear utility can make the game look nicer.
\end{description}

Note, for each function you write at each level, you should use
top-down stepwise refinement. That is to say, start with the simplest
version of the function and successively add capability to the function.
Save each program from each level.

Note also: you should see the following modules when you submit:
{\it minsweeper.c},
{\it game.c} and {\it game.h},
{\it io.c} and {\it io.h},
{\it level0.c},
{\it level1.c},
{\it level2.c},
{\it level3.c},
{\it level4.c},
and {\it level5.c}.

The {\it level} functions will be tested and graded as well as your
minesweeper program!

\end{document}
