\documentclass[12pt]{article}
\usepackage{hyperlatex}

%\usepackage{enumitem}
\usepackage{ulem}
\usepackage[compact,medium,sc]{titlesec}
\usepackage{booktabs}
\usepackage[margin=0.5in]{geometry}
\usepackage[pdftex]{graphicx}
\usepackage{multirow}
\usepackage{pslatex}
\usepackage{color}

% paragraph style

\setlength\parskip{10pt}
\setlength\parindent{0in}

\W\newcommand{\sf}{\bf}

\htmltitle{CS100: Project 2}
\htmldepth{1}
\htmladdress{\xlink{lusth@cs.ua.edu}{mailto: lusth@cs.ua.edu}}
\htmlcss{lusth.css}

\title{CS100: Project 2\\
\date{Revision Date: \today}}

\begin{document}

\maketitle

\thispagestyle{empty}

\W\subsubsection*{\xlink{Printable Version}{project2.pdf}}
\W\htmlrule

%Spring 2015

\section*{Preamble}

\begin{quote}
{\it Life is full of difficult decisions}
\end{quote}

Ever dropped food on the floor?  We're not talking about pickled brussels sprouts 
that you probably pushed off your plate on purpose when you were a child, but
good food that you really like.  Once it hits the floor, your initial instinct is
that you've got to throw it away.  However, in the back of your mind, you are
wondering if it is still good to eat.  Especially if no one else noticed that
you just picked something up from the floor.

This assignment develops a program to help one decide whether or not to eat
food that they dropped on the floor.
Using \xlink{this flowchart}{./project2flowchart.pdf},
write a program that asks the user a series of questions about the food that
was just dropped and gives them an answer -- whether or not they can still eat
what dropped on the floor.

\section*{Where to Start}

The first thing you should do is study 
\xlink{the flowchart}{./project2flowchart.pdf} and make sure you understand
the logic behind it.  The flowchart asks a series of questions and expects specific
answers to each question.  For example, it assumes you dropped one of four
possible items, {\it meat} or {\it produce} or {\it bread-roll} or {\it junk-food}.
It asks some {\it yes}/{\it no} questions and is also has a couple of routines
that it calls to find out information about when you last cleaned your floor (the options 
range {\it today} to {\it do-not-know}) and what the item you dropped looks like
after you blow on it and brush off any visible dirt and grime (the options for
this test range from {\it immaculate} to {\it still-yucky}).

Next, establish a directory in which you will work.
Run the following commands in a terminal window, one at a time:

\begin{verbatim}
    cd
    cd cs100
    mkdir -p projects/project2
    cd projects/project2
\end{verbatim}

% Jeffrey : Fixed, from {|tt eat.c} to {\tt eat.c}
Name your main module {\tt eat.c}.  You will also need to copy the {\tt scanner.h} and
{\tt scanner.c} files from {\it troll} into this directory.
You will also
create an input-output module (\verb!io.c! and \verb!io.h!)
and a logic module (\verb!logic.c! and \verb!logic.h!).

\section*{Assumptions for this Project}

We are assuming that all submitted projects will:

\begin{itemize}
\item Require that user always enters lower-case answers to the questions
\item Require that user always enters single-word answers to the questions.  There
may be special characters (such as the --) in the word, but each answer is a
single token (block of characters).
\item Use the {\tt scanner.h} and {\tt scanner.c} routines for all input
\item Use functions to break the overall project down into smaller modules
\end{itemize}

\subsection*{Functions in your Program}

At a minimum, your program should have eight functions.

\begin{itemize}
\item Four functions that handle a person dropping {\tt meat}, {\tt produce},
{\tt bread-roll}, and {\tt junk-food}
\item A {\tt YesNo} function that asks a specific question and gets a yes/no answer.
For this function, we suggest passing the specific question as a character string
to the function, and simply returning a value to determine whether the user answered
yes or no (return 1 for {\it yes} and 0 for {\it no}).
\item A {\tt FloorLastCleaned} function that asks when the floor was last cleaned, 
one of {\it today}, {\it yesterday}, {\it this-week}, or {\it do-not-know}.
\item A {\tt BlowBrushTest} function that asks how clean the food is after you
blow on it and brush off the obvious dirt, one of {\it immaculate}, {\it looks-clean},
{\it visible-specks}, or {\it still-yucky}.
\item A {\tt ThreeSecondRule} function that asks whether or not the food was on
the floor for more than three seconds before you picked it up.  Specifically, the
function should ask the user to give the number of seconds that the food has been on
the floor.  You may assume that the user will always enter a positive integer in
response to the question.  
\end{itemize}

As an example for the {\tt YesNo} function, take the situation where you are asking
whether or not the {\it donut} (type of {\it junk-food}) that dropped had sprinkles on it.
The function signature and function call shown below would be one way to approach the task.

\begin{verbatim}
    int YesNo(char *);
    ...
    int ans = YesNo("Did the donut have sprinkles on it?");
\end{verbatim}

You may introduce more functions if you wish.  For the functions listed above,
you must figure out what arguments you want to pass and how you want the results
returned.  For example, you could return the results from the {\tt YesNo} function as
an integer answer or a character (or a string or a double).  The choice is yours.

\subsection*{Program Input}

For this program, you can assume that all input is entered correctly.
All input will be lower-case responses.  The user will also give a valid response.
The complete list of input words that you might see is shown below:

{\it meat, produce, bread-roll, junk-food, fish, canned, fresh, beef, chicken,
donut, chips, yes, no, immaculate, looks-clean, visible-specks, still-yucky, today,
yesterday, this-week, do-not-know}

You will use the scanner module to perform all input.

\subsection*{Program Output}

As your program runs, it will prompt the user for information.  The user will
reply with a valid answer to each question.  After asking these questions, your
program must determine whether or not to recommend eating the dropped food.  Your
program should:

\begin{itemize}
\item Print a message to the user indicating {\bf eat} or {\bf do not eat}.
\item Exit with a return code of 0 (success) if the program recommends eating
the food and a 1 (failure) if the program recommends not to eat what dropped.
\end{itemize}

Your output will be graded based on this return code.  So the {\tt return}
statement at the end of your {\it main} routine now actually means something.
If your program exits with a {\bf return 0;} then the food is safe to eat while
exiting with a {\bf return 1;} statement is a recommendation not to eat the food.

\subsection*{Sample Execution}

The following dialog illustrates how the program might operate.  Note that 
your program's prompts are shown {\color{blue} like this} and
user responses are shown {\color{red} like this}.

{\color{blue} What fell on the floor?} {\color{red} {\it meat}}

{\color{blue} What kind of meat?} {\color{red} {\it chicken}}

{\color{blue} Chicken nuggets?} {\color{red} {\it yes}}

{\color{blue} Recommendation - you can eat what you dropped}

Another example of this program in action is:

{\color{blue} What fell on the floor?} {\color{red} {\it junk-food}}

{\color{blue} What kind of junk food?} {\color{red} {\it donut}}

{\color{blue} Did it have sprinkles on it?} {\color{red} {\it yes}}

{\color{blue} When did you last clean the floor?} {\color{red} {\it today}}

{\color{blue} What does it look like after blowing on it and brushing it off?} {\color{red} {\it looks-clean}}

{\color{blue} Recommendation - do not eat what you dropped }

The exact wording of the questions that you ask is not critical.  So whether you
say {\tt What fell on the floor?} or {\tt What did you drop on the floor?} does
not matter.  Your program must simply follow the correct paths through the
\xlink{flowchart}{./project2flowchart.pdf},
for the answers given, and then produce the correct eat-or-don't-eat answer.

\subsection*{Program Organization}

A guiding principle of 
Computer Scientists
is ``separation of concerns''. Common concerns are input-output,
logic, and data storage. By separating the concerns of the
program, it makes it easier to modify the program for different environments,
say moving your program from a desktop computer to a phone.
For this project, you will separate the input-output and logic concerns
from the main module.

All functions/procedures that call the scanner
or write/print data should be placed in
the file \verb!io.c!, with their prototypes placed in \verb!io.h!.

All functions/procedures that implement the logic of the flowchart should
be placed in the file \verb!logic.c!,
with their prototypes place in \verb!logic.h!

The main function should be placed in a file named \verb!eat.c!.

For this project, you should have a header block of comments that explains the
purpose of each module and also identifies the program author and date.  Each 
function in the program should have a short header block (one or two lines) that
explains the purpose of the function.

A good rule to remember with commenting is that if you had to sit down and think
about the code to enter in a given statement, then it probably deserves a comment.
If you had to think about the statement, then comments will help anyone who
is looking at your code at a later date.

As always, you should use consistent brace placement and indentation.
Your code should have a pleasing look.

\subsection*{A Makefile is Required}

Please make sure to generate a {\tt makefile} for this project and include it in your
submission.  Since you have four modules in this project,
{\tt eat.c}, {\tt io.c}, {\tt logic.c},  and {\tt scanner.c},
you should include a {\tt makefile} that
allows you to compile this project cleanly and easily.

Your makefile should respond to the commands:

\begin{description}
\item[make] typing \verb!make! should have your makefile build your executable
\item[make test] typing \verb!make test! should have your makefile test your
executable with a file named {\it test.dat}
\item[make clean] typing \verb!make clean! should remove the object files and the executable
\end{description}

Your makefile should never perform any unnecessary compilations.

\section*{Challenges}

Challenge One: Handle input from the user regardless of case.  That is,
for a {\bf yes} answer, the user should be able to specify
{\color{blue}{\it Yes}} or {\color{blue}{\it yes}}
or {\color{blue}{\it YES}} or any other combination of upper-and-lower-case letters.

Challenge Two: Handle the case where the user mis-spells a word.  Ask the user to re-enter
a valid response.  For example, if the user enters {\bf on} instead of {\bf no} then 
catch this error and ask them for a valid response.

\section*{Compliance Instructions}

First, check to make sure that you are using proper programming
style for this project.  The document
\xlink{style}{../style.html} contains a number of expected
conventions that must be followed for this project (as well as all
future projects).  Please make sure your program adheres to the
requirements identified in the 
\xlink{style guide}{../style.html}.

Second, create a file named {\it test.dat} that holds a proper set of responses 
for your program.

Compile and test your program, naming the executable {\it eat}, by entering
the following commands and input:

\begin{verbatim}
    $ gcc -Wall -g eat.c io.c logic.c scanner.c -o eat
    $ cat test.dat | eat
\end{verbatim}

These commands must not fail.

Third, test your makefile with the commands:

\begin{verbatim}
    $ make
    $ make test
\end{verbatim}

You should get the exact same behavior as when you manually compiled the
program and piped your {\it test.dat} file to the executable.

\section*{Checklist}

Did you:

\begin{itemize}
\item
    name your main module {\it eat.c}?
\item
    place your functions in the proper modules?
\item
    have your main function return zero or one appropriately?
\item
    comment your modules sufficiently?
\item
    create your {\it test.dat} file and test your program with it?
\item
    test your makefile thoroughly
    to ensure it builds and tests your executable?
\item
    test your makefile thoroughly
    to ensure it performs no unnecessary compilations?
\end{itemize}

Note: this list is not exhaustive! There may be other specifications
and requirements not listed here.

\section*{Submission Instructions}

Change to the {\it project2} directory containing your assignment.  Do an
{\it ls} command. You should see something like this:

{\color{red} eat.c ~~ io.c ~~ io.h ~~ logic.c ~~ logic.h ~~ makefile ~~ scanner.c ~~ scanner.h }

Extra files are OK, as long as you are in the correct directory. 
Submissions from the wrong directory
will be penalized.

Submit your program like this:

\begin{verbatim}
    submit cs100 YYY project2
\end{verbatim}

Remember to replace \verb!YYY! with your instructor name
(brown,cordes,lusth).

\section*{Due Date}

The due date for this assignment can be found on the class
\xlink{schedule}{../schedule.html}.

\end{document}
