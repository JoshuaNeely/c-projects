\documentclass{article}  

\usepackage{booktabs}
\usepackage{hyperlatex}
\usepackage[margin=0.5in]{geometry}
\usepackage{color}

\htmldepth{1}
\htmltitle{CS 150}
\htmladdress{lusth@cs.ua.edu}
\htmlcss{lusth.css}

\T\setlength\parskip{6 pt}
\T\setlength\parindent{0 in}

\title{CS 150~~Programming I\\
Project 3}

\date{Revision Date: \today}

\begin{document}

\maketitle

\thispagestyle{empty}

\W\subsubsection{\xlink{Printable Version}{project3.pdf}}
\W\htmlrule


%Fall 2008

\section*{Preamble}
This is your fourth C assignment.
You may develop your code anywhere,
but you must ensure it
runs correctly when compiled by {\it gcc}
on a Linux or Unix system
before submission.
You should read up to and include the textbook chapter
\xlink{Matrices}{../book/index_24.html}
before starting this project.


\section*{Connect-N}

Connect Four is a popular children's game developed by Milton Bradley 
and Hasbro. In this game, two players choose colors and then take turns 
dropping colored discs from the top of a suspended grid. The pieces fall 
downward, occupying the next available space within a column. The object 
of the traditional game is to connect four of one's own colored discs 
vertically, horizontally, or diagonally.

The goal of this project is to implement a variant of the traditional 
Connect Four game in C.

Tasks for your program:

\begin{itemize}
\item
    Create board
\item
    Assign players
\item
    Display board
\item
    Take moves from player or computer
\item
    Alternates between players checking for ending conditions
\item
    Displays winner
\end{itemize}


\section*{Program behavior}

Your program should take in a width and height for the board, as well as 
a value for N (the number of adjacent pieces needed to win)
from the command-line. The program should then 
prompt the user for the number of players participating in 
the game (either 1 or 2, if 1 then the computer will be the 
second player). The program should then display the board 
and prompt the player for the first move. Your display 
should look like this:

\begin{verbatim}
    [ ] [ ] [ ] [ ] [ ] [ ] [ ]
    [ ] [ ] [ ] [ ] [ ] [ ] [ ]
    [ ] [ ] [ ] [ ] [ ] [ ] [ ]
    [ ] [ ] [ ] [ ] [ ] [ ] [ ]
    [ ] [ ] [ ] [ ] [ ] [ ] [ ]
    [ ] [ ] [ ] [ ] [ ] [ ] [ ]
    [ ] [ ] [ ] [ ] [ ] [ ] [ ]
    Enter Player 1's move:
\end{verbatim}

The player will then enter an integer for the column 
where they wish to drop their piece (the valid range is 0 to the number of columns - 1). For the purposes of 
this game, you will use the symbol 'X' to refer to Player 1's 
pieces and 'O' to refer to Player 2's pieces. After the first 
move, your board should look similar to the following (depending 
on the column selected):

\begin{verbatim}
    [ ] [ ] [ ] [ ] [ ] [ ] [ ]
    [ ] [ ] [ ] [ ] [ ] [ ] [ ]
    [ ] [ ] [ ] [ ] [ ] [ ] [ ]
    [ ] [ ] [ ] [ ] [ ] [ ] [ ]
    [ ] [ ] [ ] [ ] [ ] [ ] [ ]
    [ ] [ ] [ ] [ ] [ ] [ ] [ ]
    [ ] [ ] [ ] [X] [ ] [ ] [ ]
    Enter Player 2's move:
\end{verbatim}

In the case that the computer is playing for Player 2, 
you will instead see the next prompt for Player 1. 

The game will continue in a turn-based manner until N of 
the same piece are in a row or every space on the board 
is filled (resulting in a tie). 

\subsection*{Winning Scenarios} 

The objective of the game is to be the first player to get N of 
their piece in a row. Pieces can appear horizontally:

\begin{verbatim}
    [ ] [ ] [ ] [ ] [ ] [ ] [ ]
    [ ] [ ] [ ] [ ] [ ] [ ] [ ]
    [ ] [ ] [ ] [ ] [ ] [ ] [ ]
    [ ] [ ] [ ] [ ] [ ] [ ] [ ]
    [ ] [ ] [ ] [ ] [ ] [ ] [ ]
    [ ] [ ] [ ] [O] [O] [ ] [ ]
    [X] [X] [X] [X] [O] [ ] [ ]
    Enter Player 2's move:
\end{verbatim}

Vertically:

\begin{verbatim}
    [ ] [ ] [ ] [ ] [ ] [ ] [ ]
    [ ] [ ] [ ] [ ] [ ] [ ] [ ]
    [ ] [ ] [ ] [ ] [ ] [ ] [ ]
    [ ] [ ] [ ] [ ] [O] [ ] [ ]
    [ ] [ ] [ ] [X] [O] [ ] [ ]
    [ ] [ ] [ ] [X] [O] [ ] [ ]
    [O] [X] [X] [X] [O] [X] [ ]
    Enter Player 2's move:
\end{verbatim}

Or diagonally:

\begin{verbatim}
    [ ] [ ] [ ] [ ] [ ] [ ] [ ]
    [ ] [ ] [ ] [ ] [ ] [ ] [ ]
    [ ] [ ] [ ] [ ] [ ] [ ] [ ]
    [ ] [ ] [ ] [ ] [X] [ ] [X]
    [ ] [ ] [ ] [ ] [O] [X] [O]
    [ ] [ ] [ ] [X] [X] [O] [X]
    [ ] [ ] [ ] [X] [O] [O] [O]
    Enter Player 2's move:
\end{verbatim}

\begin{verbatim}
    [ ] [ ] [ ] [ ] [ ] [ ] [ ]
    [ ] [ ] [ ] [ ] [ ] [ ] [ ]
    [ ] [ ] [ ] [ ] [ ] [ ] [ ]
    [X] [ ] [ ] [ ] [ ] [ ] [ ]
    [O] [X] [O] [ ] [ ] [ ] [ ]
    [X] [O] [X] [X] [ ] [ ] [ ]
    [O] [O] [O] [X] [ ] [ ] [ ]
    Enter Player 2's move:
\end{verbatim}

\subsection{Computer Moves}

If the player selects only 1 for the number of players, then
the computer will play the part of Player 2. The computer
will be an easy opponent. When it is the computer's turn, 
the computer will first determine which moves are legal, and
then from this list of moves, the computer will choose one
move at random to play.

The function you write for making the computer's moves should
look similar to this:

\begin{verbatim}
int computerPlayMove(Board *board, int seed)
{
	//determine the number of legal moves 

	//allocate an array to hold the legal moves

	//determine the legal moves and add them to the array

	//if only 1 legal move, then play it

	//if more, randomly select an index from the array
	
	//play the corresponding move for the index

	//return the move that the computer just played
}
\end{verbatim}

You must initialize a random number generator using srand() with a seed passed in by the user. You should do this in your main function. You can then use the rand() 
function to generate a random integer. Be careful to make sure that the number you use is not outside the bounds of your array. 

\section*{The Board}

We will be making use of structures again for this project. 
Please read the \xlink{{\it Structures}}{../book/index\_17.html}
chapter of the book before continuing.

The structure that you will need this time is as follows:

\begin{verbatim}
    typedef struct board
        {
        	int **spaces;
        	int width;
        	int height;
        } Board;
\end{verbatim}

Complete the structure and place it in a file named {\it gameio.h}.

The spaces in this structure refer to the two dimensional array of spaces
that represent the grid. The width represents the number of columns across
one row of the grid while the height represents the total number of rows
we have in the grid. 

\section*{Program organization}

Create a directory named {\it project3} that hangs off of your 
{\it cs150} directory. Do your work in that directory.

There are three main components to this project. Part of being a good
programmer is learning how to properly group components of a software
system so that parts of the program sharing similar responsibilities 
are together. For this reason, we break our Connect-N up according
to the game's logic, input and output, and control. 

Your main file will be named {\it connect.c}. Create another file called
{\it logic.c} with a header file named {\it logic.h}. You will need two more
files, {\it gameio.c} and {\it gameio.h}.

{\it logic.c} should contain the following functions:

\begin{itemize}
\item{A function that takes in a Board and an integer for a move. The function then checks whether the given move will be legal on the supplied boards. If legal, the function will return 1, otherwise it will return 0}
\item{A procedure that takes in a Board, an integer for a move, and an integer for the current player. It then sets the next free position in the column indicated by the move to be the player's}
\item{A function that takes in a game board and performs a move for the computer. This function should return the number of the column where the computer played.}
\item{A function that takes in a board and returns whether or not there are any possible moves left.}
\item{A function that takes in a board, the last player to make a move, the number of pieces that need to be connected, and the last move played. With the help of a helper function, the function then determines whether or not that player just made the winning move.}
\item{A helper function that takes in a board, the number of pieces needing to be connected, a row and a column for the board, directional modifiers for the row and column representing one of the eight possible directions located around the current space on the board (as an example in order to go up in the game grid, the row would need to be subtracted by 1 and the column would need to remain the same, therefore the values would be -1 and 0),the current player, and a count of how many of that player's pieces have been seen in the current direction. This function will use recursion to continue in one direction on the game board until either there are N pieces in a row or there are no more spaces in the current direction. It then returns the number of pieces for the given player found in that direction. Hint: be sure to increment the count each time the function recurs.:}
\end{itemize}

{\it gameio.c} should contain the following functions:

\begin{itemize}
\item{A function that gets the number of players from the user and returns it.}
\item{A function that takes in a width and height and creates a Board struct. It should allocate the two dimensional array that holds the spaces and initialize the Board struct's width and height. It should then return the newly created Board.}
\item{A procedure that initializes all spaces on the board to 0.}
\item{A procedure that takes in a board and prints it to the terminal. It checks to see which player controls which squares. Squares held by Player 1 are printed as "[X]" and squares held by Player 2 are printed as "[O]". Squares held by neither player are printed as "[ ]".}
\item{A function that prompts the player for the next move and then returns it as an integer.}
\item{A procedure that takes in an integer representing the winning player and prints a congratulatory message. }
\end{itemize}

{\it connect.c} should only contain the main function. All 
other functions should be contained within their respective files. 
Be sure to place function signatures into the appropriate header files.

\section*{Stepwise refinement}

Never try to write an entire program in one go. Always implement a little bit of what you need to do and test that bit thoroughly with a specialized main function. This technique is known as stepwise refinement.

\subsection*{Level 0}
Define a function that prompts the user for the number of players in the game (1 or 2).
Write a program named {\it level0.c} that tests this function.

\subsection*{Level 1}
Define a function that creates a board using arguments for width and height
that are passed in by the command line. You should then call the initialize procedure that sets the initial values for each spot on the board.
Please read \xlink{{\it Matrices}}{../book/index\_24.html} for additional
information.
Copy the {\it level0.c} program to {\it level1.c}.
Modify {\it level1.c} so that it tests these routines.

\subsection*{Level 2}
Define a procedure that takes in a board and displays it.
Please read \xlink{{\it Matrices}}{../book/index\_24.html} for additional
information.
Copy the {\it level1.c} program to {\it level2.c}.
Modify {\it level2.c} so that it tests these routines.

\subsection*{Level 3}
Write the function that gets the next move for the player. This function prompts the user for their next move and then returns the integer representing the column for the move. In addition, add the function that checks whether a move is legal. You should continue calling for the player's next move until the give a legal move. 
Copy the {\it level2.c} program to {\it level3.c}.
Modify {\it level3.c}.

\subsection*{Level 4}
Write the function that takes in the board, the move to be played, and the current player and plays the move for the player. Display the board after the move. 
Copy the {\it level3.c} program to {\it level4.c}.
Modify {\it level4.c} to test the new function.

\subsection*{Level 5}
Copy {\it level4.c} to {\it level5.c}.
Wrap the code that prompts the players for moves and plays them inside of a loop. For now, you can use 1 as the condition for the loop (this will create an infinite loop). You should display the board again after each move and remember to swap between players.

\subsection*{Level 6}
Add the code that plays a move for the computer. If the current player is Player 2 and there is only a single human player, have the computer make the next move. Hint: if you wish to clear the screen before displaying the board, use system("clear").
Copy {\it level5.c} to {\it level6.c}.
Modify {\it level6.c} to play the computer's moves.

\subsection*{Level 7}
Define the function that checks whether there are any moves left to play. 
Copy {\it level6.c} to {\it level7.c}.
Modify {\it level7.c} so that the loop ends when there are no moves left. You can use code similar to:

\begin{verbatim}
	int gameOver = 0;
	while(!gameOver)
	{
		//make moves
		//displayboard
		//check if there are any moves left, if not set gameOver
	}
\end{verbatim}

\subsection*{Level 8}
Define the function that checks for a winner. This function takes in the last move played and the current player and returns whether or not the current player won. This function should call a helper function that uses recursion to determine how many of the player's pieces are in a given direction (up, down, left, right, up left, up right, down left, down right). By adding numbers with the same orientation together (e.g., both up and down are vertical) check whether there are more than N pieces in a row.
Copy {\it level7.c} to {\it level8.c}.
Modify {\it level8.c} to end the game if a winner is found.

\subsection*{Level 9}

Define the function that displays the winner of the game. Feel free to make this more interesting. 
Copy {\it level7.c} to {\it connect.c}.
Modify {\it connect.c} to make sure this function works.

\section*{Program Compliance}

Your program should be able to be run as follows:

\begin{verbatim}
$connect width height n seed
How many players? 2
[ ] [ ] [ ] [ ] [ ] [ ] [ ]
[ ] [ ] [ ] [ ] [ ] [ ] [ ]
[ ] [ ] [ ] [ ] [ ] [ ] [ ]
[ ] [ ] [ ] [ ] [ ] [ ] [ ]
[ ] [ ] [ ] [ ] [ ] [ ] [ ]
[ ] [ ] [ ] [ ] [ ] [ ] [ ]
[ ] [ ] [ ] [ ] [ ] [ ] [ ]
Enter Player 1's move:
\end{verbatim}

Remember that moves should range from 0 to the number of columns - 1. In other words, the move is an index for a column.

\section*{Submission Instructions}

Change to the {\it project2} directory containing your assignment.  Use the
{\it ls} command in your projects directory. You should see something like this:

\begin{verbatim}
    connect.c   gameio.c   gameio.h   logic.c   logic.h   level0.c   level1.c   ...
\end{verbatim}

Extra files are OK. You should submit all your level files along with your main project files. Submit your project like this:

\begin{verbatim}
    submit cs150 YYY project3
\end{verbatim}

Remember to replace {\tt YYY} with your instructor name followed by class
time.

\section*{Due Date}

The due date for this assignment can be found on the class
\xlink{schedule}{../schedule.html}.

\end{document}
